%%
%% This is file `sample-sigconf-lualatex.tex',
%% generated with the docstrip utility.
%%
%% The original source files were:
%%
%% samples.dtx  (with options: `all,proceedings,bibtex,sigconf')
%% 
%% IMPORTANT NOTICE:
%% 
%% For the copyright see the source file.
%% 
%% Any modified versions of this file must be renamed
%% with new filenames distinct from sample-sigconf-lualatex.tex.
%% 
%% For distribution of the original source see the terms
%% for copying and modification in the file samples.dtx.
%% 
%% This generated file may be distributed as long as the
%% original source files, as listed above, are part of the
%% same distribution. (The sources need not necessarily be
%% in the same archive or directory.)
%%
%%
%% Commands for TeXCount
%TC:macro \cite [option:text,text]
%TC:macro \citep [option:text,text]
%TC:macro \citet [option:text,text]
%TC:envir table 0 1
%TC:envir table* 0 1
%TC:envir tabular [ignore] word
%TC:envir displaymath 0 word
%TC:envir math 0 word
%TC:envir comment 0 0
%%
%% The first command in your LaTeX source must be the \documentclass
%% command.
%%
%% For submission and review of your manuscript please change the
%% command to \documentclass[manuscript, screen, review]{acmart}.
%%
%% When submitting camera ready or to TAPS, please change the command
%% to \documentclass[sigconf]{acmart} or whichever template is required
%% for your publication.
%%
%%
\documentclass[manuscript,screen,review,anonymous]{acmart}
%%
%% \BibTeX command to typeset BibTeX logo in the docs
\AtBeginDocument{%
  \providecommand\BibTeX{{%
    Bib\TeX}}}

% For adding inline comments in the text.
\usepackage[margin=false,inline=true]{fixme}
\FXRegisterAuthor{aaa}{anaaa}{\color{cyan}AAA}
\FXRegisterAuthor{mg}{anmg}{\color{red}MG}
\FXRegisterAuthor{cz}{ancz}{\color{orange}CZ}
% \newcommand{\aaa}[1]{\aaanote{#1}}
% \newcommand{\mg}[1]{\mgnote{#1}}
% \newcommand{\cz}[1]{\cznote{#1}}
\newcommand{\aaa}[1]{}
\newcommand{\mg}[1]{}
\newcommand{\cz}[1]{}

\usepackage{stmaryrd}

\usepackage{stackengine}
\usepackage{mathrsfs}
\usepackage{braket}
\usepackage{annotate-equations}
\usepackage{scalerel}

% graphs
\usepackage{tikz}
\usetikzlibrary{arrows,automata,positioning}

% commutative diagram
\usepackage{tikz-cd}

% ref
\usepackage{hyperref}
\usepackage{cleveref}

% inference rule
\usepackage{mathpartir}
% cross-referencing infer rule
% based on https://tex.stackexchange.com/questions/340788/cross-referencing-inference-rules
\makeatletter
\let\originferrule\inferrule
\DeclareDocumentCommand \inferrule { s O {} m m}{%
  \IfBooleanTF{#1}%
  {%
    \mpr@inferstar[#2]{#3}{#4}%
  }{%
    \mpr@inferrule[#2]{#3}{#4}%
  }%
  \IfValueT{#2}%
  {%
    \my@name@inferrule{#2}%
  }%
}
\NewDocumentCommand \my@name@inferrule { m }{%
  \def\@currentlabelname{\textsc{#1}}%
}
\makeatother

% item spacing
\usepackage{enumitem}

% for code
\usepackage{listings}
\lstset{basicstyle=\footnotesize\ttfamily,breaklines=true}
\lstset{framextopmargin=50pt,frame=bottomline}
\lstset{language=caml, escapeinside={[*}{*]}}
% Clever ref names
\crefname{lstlisting}{listing}{listings}
\Crefname{lstlisting}{Listing}{Listings}

% for algorithm
\usepackage{algorithm}
\usepackage[noend]{algpseudocode}
% switch statement for pattern matching
\algnewcommand\algorithmicmatch{\textbf{match}}
\algnewcommand\algorithmicwith{\textbf{with}}
\algnewcommand\algorithmiccase{\textbf{case}}
\algnewcommand\algorithmicdefault{\textbf{default}}
\algnewcommand\Continue{\textbf{continue}}
\algdef{SE}[MATCH]{Match}{EndMatch}[1]{\algorithmicmatch\ #1\ \algorithmicwith}{\algorithmicend\ \algorithmicmatch}%
\algdef{SE}[CASE]{Case}{EndCase}[1]{\algorithmiccase\ #1 \algorithmicthen}{\algorithmicend\ \algorithmiccase}%
\algdef{SE}[DEFAULT]{Default}{EndDefault}{\hskip\algorithmicindent\algorithmicdefault}{\algorithmicend\algorithmicdefault}%
\algtext*{EndMatch}%
\algtext*{EndCase}%
\algtext*{EndDefault}%

% for better table
\usepackage{booktabs}

% subcaption
\usepackage{subcaption} %% For complex figures with subfigures/subcaptions
                        %% http://ctan.org/pkg/subcaption

% unicode math symbols
\usepackage{unicode-math}
% support for hat, overline, underline, vec, and sim combining charactors
\protected\def\afteracc{\directlua{
    local nest = tex.nest[tex.nest.ptr]
    local last = nest.tail
    if not (last and last.id == 18) then
      error'I can only put accents on simple noads.'
    end
    if last.sub or last.sup then
      error'If you want accents on a superscript or subscript, please use braces.'
    end
    local acc = node.new(21, 1)
    acc.nucleus = last.nucleus
    last.nucleus = nil
    local is_bottom = token.scan_keyword'bot' and 'bot_accent' or 'accent'
    acc[is_bottom] = node.new(23)
    acc[is_bottom].fam, acc[is_bottom].char = 0, token.scan_int()
    nest.head = node.insert_after(node.remove(nest.head, last), nil, acc)
    nest.tail = acc
    node.flush_node(last)
  }}
\AtBeginDocument{
\begingroup
  \def\UnicodeMathSymbol#1#2#3#4{%
    \ifx#3\mathaccent
      \def\mytmpmacro{\afteracc#1 }%
      \global\letcharcode#1=\mytmpmacro
      \global\mathcode#1="8000
    \else\ifx#3\mathbotaccentwide
      \def\mytmpmacro{\afteracc bot#1 }%
      \global\letcharcode#1=\mytmpmacro
      \global\mathcode#1="8000
    \fi\fi
  }
  \input{unicode-math-table}
\endgroup
}

% math font, this is needed to render \setminus command
\setmathfont{latinmodern-math}
\setmathfont[range=\setminus]{STIX Two Math}
\setmathfont[range=\similarrightarrow]{STIX Two Math}

\newtheorem{theorem}{Theorem}
\newtheorem{corollary}[theorem]{Corollary}
\newtheorem{lemma}[theorem]{Lemma}

\newtheorem{definition}{Definition}
\newtheorem*{definition*}{Definition}
\newtheorem{example}[definition]{Example}
\newtheorem{remark}[definition]{Remark}

%%%% Macros %%%%%

% Math?
\newcommand{\true}{\mathrm{true}}
\newcommand{\false}{\mathrm{false}}
\newcommand{\At}{\mathbf{At}}

% commands
\newcommand{\command}[1]{\texttt{#1}}
\newcommand{\comAssert}[1]{\command{assert}~#1}
\newcommand{\comSkip}{\command{skip}}
\newcommand{\comITE}[3]{\command{if}~#1~\command{then}~#2~\command{else}~#3}
\newcommand{\comIT}[2]{\command{if}~#1~\command{then}~#2}
\newcommand{\comWhile}[2]{\command{while}~#1~\command{do}~#2}
\newcommand{\comLabel}[1]{\command{label}~#1}
\newcommand{\comBrk}{\command{break}}
\newcommand{\comCont}{\command{continue}}
\newcommand{\comRet}{\command{return}}
\newcommand{\comGoto}[1]{\command{goto}~#1}
\newcommand{\comDup}{\command{dup}}

% operators
\newcommand{\dom}[1]{\mathrm{dom}(#1)}
\newcommand{\cod}[1]{\mathrm{cod}(#1)}
\DeclareMathOperator{\post}{\mathrm{post}}
\newcommand{\reject}{\mathinner{\mathrm{rej}}}
\newcommand{\accept}{\mathinner{\mathrm{acc}}}
\DeclareMathOperator*{\bigplus}{\scalerel*{+}{\sum}}
\newcommand{\clos}[1]{\mathrel{\overline{#1}}}
\DeclareMathOperator{\norm}{\mathrm{norm}}
\DeclareMathOperator{\dead}{\mathrm{dead}}
\DeclareMathOperator{\symb}{\mathrm{symb}}
\DeclareMathOperator{\unsymb}{\symb^{-1}}

% continuations
\newcommand{\conti}[1]{\mathbf{#1}}
\newcommand{\acc}[1]{{\conti{acc} ~ #1}}  % accept with an output indicator value
\newcommand{\setc}[1]{{\conti{set} ~ #1}}  % set variables in NetKAT
\newcommand{\retc}{{\conti{ret}}}  % return, terminate the program
\newcommand{\brkc}[1]{\conti{brk} ~ #1}  % break with an output indicator value
\newcommand{\jmpc}[1]{{\conti{jmp} ~ #1}} % goto a label

% set of models
\newcommand{\theoryOf}[1]{\ensuremath{\mathsf{#1}}}
\newcommand{\Exp}{\theoryOf{Exp}}
\newcommand{\Prog}{\theoryOf{Prog}}
\newcommand{\BExp}{\theoryOf{BExp}}
\DeclareMathOperator{\GS}{\mathrm{GS}}

\newcommand\altxrightarrow[2][0pt]{\mathrel{\ensurestackMath{\stackengine%
  {\dimexpr#1-7.5pt}{\xrightarrow{\phantom{#2}}}{\scriptstyle\!#2\,}%
  {O}{c}{F}{F}{S}}}}
\newcommand{\transvia}[1]{
    \mathrel{\raisebox{-2px}{\(\altxrightarrow[-2px]{#1}\)}}
}
\newcommand{\transOut}[2]{⇒_{#1} #2}
\newcommand{\transRej}[2]{↓_{#1} #2}

%%%%%%%%%%
%% Rights management information.  This information is sent to you
%% when you complete the rights form.  These commands have SAMPLE
%% values in them; it is your responsibility as an author to replace
%% the commands and values with those provided to you when you
%% complete the rights form.
\setcopyright{acmlicensed}
\copyrightyear{2018}
\acmYear{2018}
\acmDOI{XXXXXXX.XXXXXXX}
%% These commands are for a PROCEEDINGS abstract or paper.
\acmConference[Conference acronym 'XX]{Make sure to enter the correct
  conference title from your rights confirmation email}{June 03--05,
  2018}{Woodstock, NY}
%%
%%  Uncomment \acmBooktitle if the title of the proceedings is different
%%  from ``Proceedings of ...''!
%%
%%\acmBooktitle{Woodstock '18: ACM Symposium on Neural Gaze Detection,
%%  June 03--05, 2018, Woodstock, NY}
\acmISBN{978-1-4503-XXXX-X/2018/06}


%%
%% Submission ID.
%% Use this when submitting an article to a sponsored event. You'll
%% receive a unique submission ID from the organizers
%% of the event, and this ID should be used as the parameter to this command.
%%\acmSubmissionID{123-A56-BU3}

%%
%% For managing citations, it is recommended to use bibliography
%% files in BibTeX format.
%%
%% You can then either use BibTeX with the ACM-Reference-Format style,
%% or BibLaTeX with the acmnumeric or acmauthoryear sytles, that include
%% support for advanced citation of software artefact from the
%% biblatex-software package, also separately available on CTAN.
%%
%% Look at the sample-*-biblatex.tex files for templates showcasing
%% the biblatex styles.
%%

%%
%% The majority of ACM publications use numbered citations and
%% references.  The command \citestyle{authoryear} switches to the
%% "author year" style.
%%
%% If you are preparing content for an event
%% sponsored by ACM SIGGRAPH, you must use the "author year" style of
%% citations and references.
%% Uncommenting
%% the next command will enable that style.
%%\citestyle{acmauthoryear}


%%
%% end of the preamble, start of the body of the document source.
\begin{document}

%%
%% The "title" command has an optional parameter,
%% allowing the author to define a "short title" to be used in page headers.
\title{SyGO: Generic On-the-fly Symbolic Algorithms For Variants of GKAT}

%%
%% The "author" command and its associated commands are used to define
%% the authors and their affiliations.
%% Of note is the shared affiliation of the first two authors, and the
%% "authornote" and "authornotemark" commands
%% used to denote shared contribution to the research.
\author{TODO}
% \author{Ben Trovato}
% \authornote{Both authors contributed equally to this research.}
% \email{trovato@corporation.com}
% \orcid{1234-5678-9012}
% \author{G.K.M. Tobin}
% \authornotemark[1]
% \email{webmaster@marysville-ohio.com}
% \affiliation{%
%   \institution{Institute for Clarity in Documentation}
%   \city{Dublin}
%   \state{Ohio}
%   \country{USA}
% }

% \author{Lars Th{\o}rv{\"a}ld}
% \affiliation{%
%   \institution{The Th{\o}rv{\"a}ld Group}
%   \city{Hekla}
%   \country{Iceland}}
% \email{larst@affiliation.org}

% \author{Valerie B\'eranger}
% \affiliation{%
%   \institution{Inria Paris-Rocquencourt}
%   \city{Rocquencourt}
%   \country{France}
% }

% \author{Aparna Patel}
% \affiliation{%
%  \institution{Rajiv Gandhi University}
%  \city{Doimukh}
%  \state{Arunachal Pradesh}
%  \country{India}}

% \author{Huifen Chan}
% \affiliation{%
%   \institution{Tsinghua University}
%   \city{Haidian Qu}
%   \state{Beijing Shi}
%   \country{China}}

% \author{Charles Palmer}
% \affiliation{%
%   \institution{Palmer Research Laboratories}
%   \city{San Antonio}
%   \state{Texas}
%   \country{USA}}
% \email{cpalmer@prl.com}

% \author{John Smith}
% \affiliation{%
%   \institution{The Th{\o}rv{\"a}ld Group}
%   \city{Hekla}
%   \country{Iceland}}
% \email{jsmith@affiliation.org}

% \author{Julius P. Kumquat}
% \affiliation{%
%   \institution{The Kumquat Consortium}
%   \city{New York}
%   \country{USA}}
% \email{jpkumquat@consortium.net}

%%
%% By default, the full list of authors will be used in the page
%% headers. Often, this list is too long, and will overlap
%% other information printed in the page headers. This command allows
%% the author to define a more concise list
%% of authors' names for this purpose.
\renewcommand{\shortauthors}{Trovato et al.}

%%
%% The abstract is a short summary of the work to be presented in the
%% article.
\begin{abstract}
  TODO
\end{abstract}

%%
%% The code below is generated by the tool at http://dl.acm.org/ccs.cfm.
%% Please copy and paste the code instead of the example below.
%%
\begin{CCSXML}
<ccs2012>
 <concept>
  <concept_id>00000000.0000000.0000000</concept_id>
  <concept_desc>Do Not Use This Code, Generate the Correct Terms for Your Paper</concept_desc>
  <concept_significance>500</concept_significance>
 </concept>
 <concept>
  <concept_id>00000000.00000000.00000000</concept_id>
  <concept_desc>Do Not Use This Code, Generate the Correct Terms for Your Paper</concept_desc>
  <concept_significance>300</concept_significance>
 </concept>
 <concept>
  <concept_id>00000000.00000000.00000000</concept_id>
  <concept_desc>Do Not Use This Code, Generate the Correct Terms for Your Paper</concept_desc>
  <concept_significance>100</concept_significance>
 </concept>
 <concept>
  <concept_id>00000000.00000000.00000000</concept_id>
  <concept_desc>Do Not Use This Code, Generate the Correct Terms for Your Paper</concept_desc>
  <concept_significance>100</concept_significance>
 </concept>
</ccs2012>
\end{CCSXML}

\ccsdesc[500]{Do Not Use This Code~Generate the Correct Terms for Your Paper}
\ccsdesc[300]{Do Not Use This Code~Generate the Correct Terms for Your Paper}
\ccsdesc{Do Not Use This Code~Generate the Correct Terms for Your Paper}
\ccsdesc[100]{Do Not Use This Code~Generate the Correct Terms for Your Paper}

%%
%% Keywords. The author(s) should pick words that accurately describe
%% the work being presented. Separate the keywords with commas.
\keywords{Do, Not, Us, This, Code, Put, the, Correct, Terms, for,
  Your, Paper}
%% A "teaser" image appears between the author and affiliation
%% information and the body of the document, and typically spans the
%% page.

\received{20 February 2007}
\received[revised]{12 March 2009}
\received[accepted]{5 June 2009}

%%
%% This command processes the author and affiliation and title
%% information and builds the first part of the formatted document.
\maketitle

\section{Introduction}

\section{Overview}

\section{GKAT with Output}

Symbolic GKAT with Output (SyGO) automata extends GKAT automata, where instead of only emitting accepts, it allows to produce a finite set of outputs \(O\).
In the case of GKAT and Guarded NetKAT, the output set \(O\) is a singleton set \(\acc{}\); whereas in CF-GKAT, the outcome can be any continuations.

Formally, a SyGO Automata is defined over a set of tests \(\BExp\), a set of outputs \(O\), and a set of state \(S\);  
with a start state \(ŝ ∈ S\), and two functions:
\begin{align*}
  ϵ & : S → 𝒫(\BExp × O) & 
  δ & : S → 𝒫(\BExp × S × Σ).
\end{align*}
Intuitively, \(ϵ(s, o)\) returns the condition required to immediately return the output \(o\) when at the state \(s\); and if \((b, s') ∈ δ(s)\) then it means that from state \(s\), when condition \(b\) is met, the machine will transition to \(s'\).

To ensure the determinism of SyGO automata, we will impose a \emph{disjointness condition}, i.e. all the conditions starting from \(s\) will be disjoint.
Formally, fix any \(s ∈ S\):
\begin{itemize}
  \item if \((b₁, o₁) ∈ ϵ(s)\) and \((b₂, o₂) ∈ ϵ(s)\), then \(o₁ ≠ o₂\) implies \(b₁\) is disjoint with \(b₂\);
  \item if \((b₁, s₁, p) ∈ δ(s)\) and \((b₂, s₂, q) ∈ δ(s)\), then \(s₁ ≠ s₂\) or \(p ≠ q\) implies \(b₁\) is disjoint with \(b₂\);
  \item for all \(ϵ(s, o)\) is disjoint with all \(b\) s.t. there exists \(s'\) where \((b, s') ∈ δ(s)\).
\end{itemize}

We list our bisimulation algorithm in~\Cref{alg:symb-bisim}.

\begin{algorithm*}
  \caption{Symbolic On-the-fly Bisimulation Algorithm}\label{alg:symb-bisim}
  \begin{algorithmic}
      \Function{equiv}{$s ∈ S, u ∈ U$}
      \If {\Call{eq}{$s, u$}} {\Return true}
      \ElsIf {\Call{knownDead\(_S\)}{$s$}} {\Return \Call{isDead\(_U\)}{$u$}} 
      \ElsIf {\Call{knownDead\(_U\)}{$u$}} {\Return \Call{isDead\(_S\)}{$s$}} 
      \Else {}
      \Return {
          %\Comment{conditions of ~\Cref{thm:symb-recursive-construction}}
          \\\vspace{5px}
          \(\qquad
          \begin{aligned}
              & ∀ o ∈ O, ϵ(s, o) ≡ ⋁ ϵ(u, o) \mathrel{\&\!\&} \\  
              & s \transvia{b ∣ p} s' \text{ and } (b ∧ ρ_U(u)) ≢ 0 \text{ implies \Call{isDead$_S$}{$s'$}} \mathrel{\&\!\&}\\
              & u \transvia{b ∣ p} u' \text{ and } (ρ_S(s) ∧ b) ≢ 0 \text{ implies \Call{isDead$_U$}{$u'$}} \mathrel{\&\!\&}\\
              & s \transvia{b ∣ p} s' \text{ and } u \transvia{c ∣ p} u' \text{ implies } \text{\Call{Union}{$s$, $u$}}; \text{\Call{Equiv}{$s', u'$}} \mathrel{\&\!\&}\\
              & s \transvia{b ∣ p} s' \text{ and } u \transvia{c ∣ q} u' \text{ and } p ≠ q \text{ implies \Call{isDead}{$s'$} and \Call{isDead}{$u'$}}
          \end{aligned}\)
      }
      \EndIf
      \EndFunction
  \end{algorithmic}
\end{algorithm*}


\section{Guarded NetKAT}

Guarded NetKAT~\cite{wasserstein_GUARDEDNETKATSOUNDNESS_2023} is a variant of NetKAT that disallows non-deterministic assignments. 
For Guarded NetKAT, the possible outputs \(O\) are \emph{partial assignments} \(σ: X ↛ V\), which are partial functions that encodes all the \emph{known} variable assignment at this stage of the execution.
And we can extend \(σ: X ↛ V\):
\[σ[x ↦ v](y) ≜ \begin{cases}
  v & \text{if } x = y \\  
  σ(x) & \text{otherwise}
\end{cases}\]

Here is the syntax for Guarded NetKAT over a set of values \(V\) and variable \(X\):
\begin{align*}
  \BExp ∋ b, c & ≜ 
    \false 
    ∣ \true 
    ∣ {\color{blue} x = v~(x ∈ X, v ∈ V)} 
    ∣ b ∧ c 
    ∣ b ∨ c 
    ∣ ¬ b 
    ∣ {\color{blue} b[σ]}\\
  \Exp ∋ e, f & ≜ 
    b ∈ \BExp
    ∣ p ∈ \{\comDup\}
    ∣ {\color{blue}x := v~(x ∈ X, v ∈ V)}
    ∣ e +_b f 
    ∣ e; f 
    ∣ e^{(b)}
\end{align*}
Note that we added a boolean expression \(b[σ]\), but following the syntax of original Guarded NetKAT~\cite{wasserstein_GUARDEDNETKATSOUNDNESS_2023}, we disallow this construct from NetKAT programs. 
\(b[σ]\) generates a new predicate by plugging in the value of \(σ: X ↛ V\) into the boolean expression \(b\).
Formally \(b[σ]\) is defined inductively as follows:
\begin{align*}
  \false[σ] & ≜ \false & 
  \true[σ] & ≜ \true \\
  (x = i)[σ] & ≜ \begin{cases}
    x = i & x ∉ \dom{σ} \\
    \true & σ(x) = i \\  
    \false & σ(x) ≠ i 
  \end{cases} & 
  (b ∧ c)[σ] & ≜ b[σ] ∧ c[σ] \\
  (b ∨ c)[σ] & ≜ b[σ] ∨ c[σ] &
  (¬ b)[σ] & ≜ ¬ (b[σ])
\end{align*}

The state in the symbolic Guarded NetKAT automata has two components: \(σ: X ↛ V\) indicating the current known assignment to variables, and \(e: \Exp\) representing the rest of the expression
\begin{mathpar}
  \inferrule
  {\\}
  {(σ, \comAssert{b})\transOut{b[σ]}{σ}}
  \and  
  \inferrule
  {\\}
  {(σ, x := v)\transOut{\true}{σ[x ↦ v]}}
  \and  
  \inferrule
  {\\}
  {(σ,\comDup)\transvia{b_σ ∣ \comDup} (σ,\comSkip)}
  \and  
  \inferrule
  {(σ, e) \transOut{c}{σ'}}
  {(σ, e +_b f) \transOut{b[σ] ∧ c}{σ'}}
  \and  
  \inferrule
  {(σ, f) \transOut{c}{σ'}}
  {(σ, e +_b f) \transOut{¬b[σ] ∧ c}{σ'}}
  \and
  \inferrule
  {(σ, e) \transvia{c ∣ p} (σ', e')}
  {(σ, e +_b f) \transvia{b[σ] ∧ c ∣ p} (σ', e')}
  \and  
  \inferrule
  {(σ, f) \transvia{c ∣ p} (σ', f')}
  {(σ, e +_b f) \transvia{b[σ] ∧ c ∣ p} (σ', f')}
  \and  
  \inferrule
  {(σ₁, e) \transOut{b}{σ₂} \\ (σ₂, f) \transOut{c}{σ₃}}
  {(σ₁, e; f) \transOut{b ∧ c}{σ₃}}
  \and  
  \inferrule
  {(σ₁, e) \transOut{b}{σ₂} \\ (σ₂, f) \transvia{b ∣ p} (σ₃, f')}
  {(σ₁, e; f) \transvia{b ∧ c ∣ p} (σ₃, f')}
  \and  
  \inferrule
  {(σ, e) \transvia{b ∣ p} (σ', e')}
  {(σ, e;f) \transvia{b ∣ p} (σ', e';f)}
  \and  
  \inferrule
  {\\}
  {(σ, e^{(b)}) \transOut{¬b[σ]}{σ}}
  \and  
  \inferrule
  {(σ₁, e) \transOut{c}{σ₂} \\  (σ₂, e^{(b)}) \transOut{a}{σ₃}}
  {(σ₁, e^{(b)}) \transOut{b[σ] ∧ c ∧ a}{σ₃}}
  \and  
  \inferrule
  {(σ₁, e) \transOut{c}{σ₂} \\  (σ₂, e^{(b)}) \transvia{a ∣ p} (σ₃, f)}
  {(σ₁, e^{(b)}) \transvia{b[σ] ∧ c ∧ a ∣ p} (σ₃, f)}
  \and  
  \inferrule
  {(σ, e) \transvia{c ∣ p} (σ', e')}
  {(σ, e^{(b)}) \transvia{b[σ] ∧ c ∣ p} (σ', e'; e^{(b)})}
\end{mathpar}
Notice that when \((σ₁, e) \transOut{c}{σ₂}\), the derivative rule with the source \((σ₁, e^{(b)})\) appears to be self-referential. 
However, since there are only finitely many partial assignments \(σ: X ↛ V\), for any finite set \(X\) and \(V\); there are only finitely many cases to check.
Practically, we use the following recursive algorithm to compute \(ϵ(σ, e^{(b)})\), where \(Ψ\) denotes the set of all explored \(σ\).
\begin{align*}
  ϵ̂(σ, e^{(b)}, Ψ) & ≜ \begin{cases}
    ∅ & \text{if } σ ∈ Ψ \\  
    \begin{aligned}
      & \{(¬b[σ], σ)\} ∪ \\
      & \{(b[σ] ∧ c ∧ a, σ₂) ∣ (a, σ₂) ∈ ϵ̂(σ₁, e^{(b)}, Ψ ∪ \{σ\})\}
    \end{aligned}
     & \text{else, if } (σ, e)\transOut{c}{σ₁}\\
  \end{cases}\\
  ϵ(σ, e^{(b)}) & ≜ ϵ̂(σ, e^{(b)}, ∅)
\end{align*}


\section{Control-Flow GKAT}

\subsection{Syntax}

Control-Flow GKAT (or CF-GKAT)~\cite{zhang_CFGKATEfficientValidation_2025} extends Guarded Kleene Algebra with tests with common control-flow operators, including indicator variable, break, continue, and goto's. 

Similar to GKAT, the syntax of CF-GKAT is also two sorted, we highlight the newly added :
\begin{align*}
  \BExp ∋ b, c ≜{}&
  \false
  ∣ \true
  ∣ t  ∈  T
  ∣ {\color{blue}x = i\ (i ∈ I)}
  ∣ b  \lor  c
  ∣ b  \land  c
  ∣ ¬ b \\
  \Exp ∋ e, f ≜{}&
    \comAssert{b}
    ∣ p  ∈  Σ
    ∣ {\color{blue}x := i\ (i ∈ I)}
    ∣ {\color{blue}e ⋆ f}
    ∣ e; f
    ∣ e +_b f
    ∣ \\
  &
    e^{(b)}
    ∣ {\color{blue} \comBrk}
    ∣ {\color{blue} \comCont}
    ∣ {\color{blue} \comRet}
    ∣ {\color{blue} \comGoto{l}\ (l ∈  L)}
    ∣ {\color{blue} \comLabel{l}\ (l ∈ L)}
\end{align*}
Additionally, for a CF-GKAT expression \(e\) to be a CF-GKAT \emph{program} (or \emph{well-formed expressions}), we enforce the following conditions:
\begin{itemize}
  \item If \(\comLabel{l}\) does not occur in \(e\), then so is \(\comGoto{l}\).
  \item Each \(\comLabel{l}\) only appear once in the program \(e\).
  \item No \(\comBrk\) or \(\comRet\) outside a while-loop.  
  \item All programs need to end with a return, i.e. \(e\) need to be of the form \(f; \comRet\) where \(f ∈ \Exp\).
  \item The \(⋆\) operator cannot appear in the program. 
\end{itemize}
In addition to the syntax introduced in the original paper~\cite{zhang_CFGKATEfficientValidation_2025}, we added two additional constructions: \(\comCont\), a common control-flow manipulating structure; and the \emph{unfolding star} \(e ⋆ f\) is an expression that is generated from loop unfolding, hence cannot appear in any CF-GKAT programs.
We use the common shorthand \(\comSkip\) for \(\comAssert{\true}\). On the other hand, \(\comAssert{\false}\) is the divergent or constant rejection program; in fact, it is equivalent to \(\comSkip^{(\true)}\).

\subsection{Label Extraction}

First, we define the label extraction process, where we extract the meaning of each label syntactically: given a label \(l ∈ L\), \((-)_l\) will take an input program, and give the expression that corresponds to the meaning of \(l\) in the input program:
\begin{align*}
  (-)_{l}&: \Prog → \Exp \\  
  (\comLabel{l'})_{l} & ≜ \begin{cases}
    \comSkip & l = l' \\  
    \comAssert{\false} & l ≠ l'
  \end{cases} & 
  (e +_b f)_{l} & ≜ \begin{cases}
    (e)_{l} & \text{if \(\comLabel{l}\) occurs in \(e\)}\\
    (f)_{l} & \text{if \(\comLabel{l}\) occurs in \(f\)}
  \end{cases} \\  
  (e; f)_{l} & ≜ \begin{cases}
    (e)_{l}; f & \text{if \(\comLabel{l}\) occurs in \(e\)}\\
    (f)_{l} & \text{if \(\comLabel{l}\) occurs in \(f\)}
  \end{cases} & 
  (e ⋆ f)_{l} & ≜ \begin{cases}
    (e)_{l} ⋆ f & \text{if \(\comLabel{l}\) occurs in \(e\)}\\
    (f)_{l} & \text{if \(\comLabel{l}\) occurs in \(f\)}
  \end{cases} \\
  (e^{(b)})_{l} & ≜ 
    (e)_l ⋆ e^{(b)}
\end{align*}
And the label extraction \((e)_l\) for every the other program \(e\) is defined to be \(\comAssert{\false}\).
Notice the \(⋆\) operation is produced in the loop case, this serves as an example of loop unfolding.

\subsection{Boolean Semantics}

For a boolean expression \(b ∈ \BExp\), the semantics is 

\subsection{Derivatives and Loop Resolution}

The symbolic derivatives of CF-GKAT can be defined as follows:




\bibliographystyle{ACM-Reference-Format}
\bibliography{refs}
\end{document}
\endinput
%%
%% End of file `sample-sigconf-lualatex.tex'.
